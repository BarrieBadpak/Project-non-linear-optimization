\documentclass[a4paper,10pt]{article}
\usepackage[latin1]{inputenc}
\usepackage{amsmath,amssymb,amscd,latexsym,dsfont}
\usepackage[english]{babel}
\usepackage{tabularx,multirow}
\usepackage{graphicx,color,array}
\usepackage{float}
\usepackage{comment,psfrag,cite}
\usepackage[top=1.75cm,bottom=1.75cm,left=2.0cm,right=2.0cm,footskip=1cm,headheight=3ex,headsep=0.3cm]{geometry}
%\usepackage{showframe}
\usepackage[small]{titlesec}
\usepackage{tikz}
\usepackage{pgfplots}
\newcommand{\GL}[1]{{\color{PineGreen} #1}}

\newcommand{\ie}{i.e.,~}
\newcommand{\eg}{e.g.,~}
\newcommand{\cf}{cf.~}

\newcommand{\tr}[1]{\mathrm{#1}}

\newcommand{\mr}[3]{\multirow{#1}{#2}{#3}}
\newcommand{\pending}[1]{{\color{red}{#1}}}

%
\usepackage{xcolor}
\usepackage{hyperref}
\usepackage{enumerate}
\hypersetup{colorlinks=true,urlcolor=cyan}

\usepackage{fancyhdr}
\chead{}
\rhead{\includegraphics[width=1.1cm]{Tue_logo.png}}
\lhead{\today}
\cfoot{\thepage}

  %% 1Col Bibliography
\let\oldthebibliography=\thebibliography
  \let\endoldthebibliography=\endthebibliography
  \renewenvironment{thebibliography}[1]%
  {\scriptsize \renewcommand{\baselinestretch}{0.9}%
    \begin{oldthebibliography}{#1}%
    \setlength{\parskip}{0pt}\setlength{\itemsep}{1ex}%
  }%
  {\end{oldthebibliography}}

\begin{document}
\pagestyle{fancy}

\title{Project non-linear optimization: optimized current input design
}
\author{J.C.P. Hakvoort \& L.M. Verhage\\
        ID: 1003407 \& 1021437 \\
        Project group number 24} 


\maketitle
\thispagestyle{empty}

\section{Question 1} \label{sec:question1}
The first question was to find a current vector $i$ such that $w=w_{desired}$ at a particular coordinate $p_x$. We invoke our sense for creativity and choose $p_x=24$ mm, the group number. We substitute the values for $p_x$ and $p_z$ to construct the coupling matrix for the particular position. Equation (\ref{eq:subsgamma}) displays the calculated coupling matrix rounded to two decimals.
\begin{equation} \label{eq:subsgamma}
    \Gamma(24,1)=
\begin{bmatrix}
2.76    & -20.33    & 17.56     & 2.76      & -20.33    & 17.56     & 2.76 \\
-21.87  & 8.54      & 13.33     & -21.87    & 8.54      & 13.33     & -21.87 \\
2715.13 & -794.97   & -746.70   & 527.74    &	59.42   & 586.31    & -1659.66 
\end{bmatrix}
\end{equation}

Next, we bluntly solve the linear system to find an initial solution for the current using our unlicensed Matlab, knowing that it is under-determined. Let us call this solution $$i_0 = \Gamma(24,1)\backslash w_{desired} = \begin{bmatrix} -0.92& -1.46& 0& 0& 0& 0& -0.80 \end{bmatrix} ^T$$. A simple check $w=\Gamma(24,1)i_0$ yields again $w_{desired}$ and we are happy.

\section{Question 2} \label{sec:question2}
Not yet sure how to proof this formally? Can do it numerically. Probably the rank of this matrix will remain full rank (=3) always, thus you are always able to find a solution.

\section{Question 3} \label{sec:question3}
We continue from the answer found in Question \ref{sec:question1}. The linear systems contains seven variables and only 3 linear independent equations, which will result in four 'free' variables. Explicitly, these solutions can be found by calculating the null-space of $\Gamma(24,1)$. Each column in equation (\ref{eq:null-space-gamma}) is a basis vector in the null-space. The null-space of $\Gamma(24,1)$ is found to be

\begin{equation} \label{eq:null-space-gamma}
N(\Gamma(24,1))=
\begin{bmatrix}
   -0.4558 &   0.2578 &   0.0373 &  -0.1561\\
   -0.2661 &  -0.0911 &   0.5197 &  -0.5690\\
   -0.0811 &   0.5494 &  -0.2216 &  -0.5409\\
    0.7448 &   0.0764 &   0.2374 &  -0.3646\\
    0.2369 &   0.7227 &   0.0068 &   0.2995\\
    0.0520 &   0.0823 &   0.7481 &   0.2713\\
   -0.3181 &   0.2974 &   0.2518 &   0.2511
\end{bmatrix}
\end{equation}

Let us again refer to our favourite number and choose our new free variable to be the $24^{th}$ letter of the alphabet, $x\in \mathbb{R}^4$. All solutions are now given by
\begin{equation} \label{eq:allsolutions}
    i(x) = i_0 + N(\Gamma(24,1))x
\end{equation}
and notice that there are infinitely many solutions. 

\noindent \textbf{Proof}
We want to proof that $i(x)$ satisfies $\Gamma(24,1)i(x) = w_{desired}$ \hspace{1ex} $\forall x$. 
\begin{align}
    \Gamma(24,1)i(x) &= w_{desired} \\
    \Gamma(24,1)(i_0 + N(\Gamma(24,1))x) &= w_{desired} \\
    \Gamma(24,1)i_0 + \Gamma(24,1)N(\Gamma(24,1))x &= w_{desired} \label{eq:proof1}
\end{align}
Now, by definition the null-space is spanned by vectors $x$ that satisfy $Ax=0$. $N(\Gamma(24,1))x$ is a linear combination of these vectors and hence the term $\Gamma(24,1)N(\Gamma(24,1))x$ in (\ref{eq:proof1}) equals zero. The resulting expression $\Gamma(24,1)i_0= w_{desired}$ was proven to be a valid solution in question \ref{sec:question1}. This concludes the proof.

\section{Question 4} \label{sec:question4}
We now formulate the problem as an optimization problem.

\begin{equation}
\begin{array}{ll@{}ll}
\text{minimize}  & P(i)=i(x)^TRi(x)\\
\text{subject to}& \Gamma(p_x,p_z)i(x)=w_{desired}
\end{array}
\end{equation}
Note that, all solutions from question \ref{sec:question3} are parameterized in $x$ and they satisfy the constraint $\Gamma(p_x,p_z)i(x)=w_{desired} \forall x$. Therefore, the problem is an unconstrained optimization problem, since we can choose $x \in \mathbb(R)^4$. 

\section{Question 5}
\cite{minimax}

\bibliographystyle{IEEEtran}
\bibliography{bib}

\end{document}

